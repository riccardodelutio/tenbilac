% debut d'un fichier latex standard
\documentclass[a4paper,12pt,twoside]{article}

% pour l'inclusion de figures en eps,pdf,jpg
\usepackage{graphicx}
\usepackage{epstopdf}
% quelques symboles mathematiques en plus
\usepackage{listings}
\usepackage{amsmath}
\usepackage{morefloats}
\usepackage{verbatim}
% le tout en langue francaise
%\usepackage[francais]{babel}
% on peut ecrire directement les caracteres avec l'accent
% a utiliser sur Linux/Windows
\usepackage[utf8]{inputenc}
\usepackage[squaren,Gray]{SIunits}
\usepackage{subfigure}
\usepackage{color}
\usepackage{multirow}
% a utiliser sur le Mac
%\usepackage[applemac]{inputenc}
% pour l'inclusion de links dans le document 
\usepackage[colorlinks,bookmarks=false,linkcolor=blue,urlcolor=blue]{hyperref}
\usepackage{wrapfig}
\paperheight=297mm
\paperwidth=210mm

% configuration listings (insertion de code)
\definecolor{colKeys}{rgb}{0,0,0.8}
\definecolor{colIdentifier}{rgb}{0,0,0}
\definecolor{colComments}{rgb}{0.6,0.6,0.6}
\definecolor{colString}{rgb}{0.8,0,0}
\lstset{
float=hbp,
basicstyle=\ttfamily\small,
identifierstyle=\color{colIdentifier},
keywordstyle=\color{colKeys},
stringstyle=\color{colString},
commentstyle=\color{colComments},
columns=flexible,
tabsize=2,
frame=tRBl,
frameround=tttt,
extendedchars=true,
showspaces=false,
showstringspaces=false,
numbers=left,
numberstyle=\tiny,
breaklines=true,
breakautoindent=true,
captionpos=b,
xrightmargin=-1cm,
xleftmargin=-1cm
}

\setlength{\textheight}{235mm}
\setlength{\topmargin}{-1.2cm} % pour centrer la page verticalement
%\setlength{\footskip}{5mm}
\setlength{\textwidth}{15cm}
\setlength{\oddsidemargin}{0.56cm}
\setlength{\evensidemargin}{0.56cm}
\DeclareMathOperator{\sech}{sech}
\pagestyle{plain}

% quelques abreviations utiles
\def \be {\begin{equation}}
\def \ee {\end{equation}}
\def \dd  {{\rm d}}


\newcommand{\mail}[1]{{\href{mailto:#1}{#1}}}
\newcommand{\ftplink}[1]{{\href{ftp://#1}{#1}}}


\begin{document}

 % Le titre, l'auteur et la date
\title{Figuring out Back-propagation}
\maketitle

\section{Conventions}
\begin{itemize}
\item Weight from node $j$ to node $i$ in the $l$ layer : $W^{(l)}_{ij}$
\item Bias to node $i$ in the $l$ layer : $b_i^{(l)}$
\item $j$-th input of layer $l$ : $x_j^{(l)~r,c}$
\item Activation of node $i$ in the layer $l$ : $a_i^{(l)~r,c}$
\item Total weighted sum including the bias : $z_i^{(l)~r,c}$
\item Hypothesis or output $i$ of the network : $h_{i}^{r,c}(x^{r,c}_i) = a^{(out)~r,c}_i$
\item Target $i$ of the network : $y_i$
\item "Error term" of node $i$ in layer $l$ : $\delta_i^{(l)~r,c}$
\item Activation function $f$ : $a_i^{(l)~r,c}=f(z_i^{(l)~r,c})=\tanh(z_i^{(l)~r,c})$
\item Derivative of the activation function : $f'(z_i^{(l)~r,c})=\sech(z_i^{(l)~r,c})^2$
\item Activation function of the output layer $f^{(out)}$ : \\$h_{i}^{r,c}(x^{r,c}_i)=f^{(out)}(z_i^{(out)~r,c})=z_i^{(l)~r,c}$
\item Derivative of the activation function : $f'(z_i^{(out)~r,c})=1$
\item Cost function, here we use the Sum Square Bias : \\ $J(W,b) = \frac{1}{2} \sum_i \sum_c \left( \frac{\sum_r h_{i}^{r,c}}{\#r} - y_i \right)^2$
\item $r$ denotes the realization and $c$ the case
\end{itemize}

\section{Adapting back-propagation to our case}

The goal is to change our weights so that we lower the value of the cost function. This can be done by gradient descent : 

\begin{equation}
\Delta W_{ij}^{(l)} = -\eta \frac{\partial J(W,b)}{\partial W_{ij}^{(l)}}~,
\label{graddescent}
\end{equation}

where $\Delta W_{ij}^{(l)}$ is the amount by which we want to change the weight $W_{ij}^{(l)}$, and where $\eta$ is an arbitrarily chosen parameter (ideally this won't be the case) and is generally $\sim 10^{-3}$. Back-propagation is a way of calculating $\frac{\partial J(W,b)}{\partial W_{ij}^{(l)}}$.\\
Using the chain rule we have :

\begin{equation}
\frac{\partial J(W,b)}{\partial W^{(l)}_{ij}} =\sum_{r,c} \frac{\partial J(W,b)}{\partial z^{(l)~r,c}_{i}} \frac{\partial z^{(l)~r,c}_{i}}{\partial W^{(l)}_{ij}} = \sum_{r,c} \frac{\partial J(W,b)}{\partial z^{(l)~r,c}_{i}}x_j^{(l)~r,c}~.
\label{costder}
\end{equation}
\\
Let's define the "error terms" by : $\delta^{(l)~r,c}_i = \frac{\partial J(W,b)}{\partial z^{(l)~r,c}_i}$. \\
Thus we can rewrite eq.\ref{costder} as follows :

\begin{equation}
\frac{\partial J(W,b)}{\partial W^{(l)}_{ij}} =\sum_{r,c} \delta^{(l)~r,c}_i x_j^{(l)~r,c}~.
\end{equation}
\\
Let's now find an easy way to calculate the "error terms", starting by the output layer : 

\begin{equation}
\delta^{(out)~r,c}_i  = \frac{\partial J(W,b)}{\partial z^{(out)~r,c}_i} = \frac{\partial J(W,b)}{\partial a^{(out)~r,c}_i} \frac{\partial a^{(out)~r,c}_i}{\partial z^{(out)~r,c}_i} = \frac{\partial J(W,b)}{\partial a^{(out)~r,c}_i} f'^{(out)}(z^{(out)~r,c}_i) = \frac{\partial J(W,b)}{\partial h^{r,c}_i}~,
\label{deltaout}
\end{equation}

because the activation function for the output layer is the identity function (derivative of 1) and $a^{(out)~r,c}_i = h^{r,c}_i$ by definition.\\
Using the definition of the cost function we obtain : 

\begin{equation}
\frac{\partial J(W,b)}{\partial h^{r,c}_i}  = \frac{1}{\# r}\left(\frac{\sum_r h_i^{r,c}}{\# r} - y_i \right)~.
\end{equation}
 \\

Thus we can rewrite eq.\ref{deltaout} as follows : 

\begin{equation}
\delta^{(out)~r,c}_i = \frac{1}{\# r}\left(\frac{\sum_r h_i^{r,c}}{\# r} - y_i \right)~.
\end{equation}
\\

And now for the deltas of the other layers we have :

\begin{equation}
\delta^{(l)~r,c}_i = \frac{\partial J(W,b)}{\partial z^{(l)~r,c}_i} = \sum_k \frac{\partial J(W,b)}{\partial z^{(l+1)~r,c}_k}\frac{\partial z^{(l+1)~r,c}_k}{\partial z^{(l)~r,c}_i} = \frac{\partial J(W,b)}{\partial z^{(l+1)~r,c}_k}\frac{\partial z^{(l+1)~r,c}_k}{\partial a^{(l)~r,c}_i} \frac{\partial a^{(l)~r,c}_i}{\partial z^{(l)~r,c}_i} 
\end{equation}
\[
= f'(z^{(l)~r,c}_i) \sum_k \delta^{(l+1)~r,c}_k W^{(l+1)}_{ki}~,
\]
where $k$ denote the indices of the nodes in layer $(l+1)$.

\end{document}